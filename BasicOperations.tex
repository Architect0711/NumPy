\subsection{Basic Operations}
Arithmetic operators on arrays apply elementwise. A new array is created and filled with the result.
>>> a = array( [20,30,40,50] )
>>> b = arange( 4 )
>>> b
array([0, 1, 2, 3])
>>> c = a-b
>>> c
array([20, 29, 38, 47])
>>> b**2
array([0, 1, 4, 9])
>>> 10*sin(a)
array([ 9.12945251, -9.88031624,  7.4511316 , -2.62374854])
>>> a<35
array([True, True, False, False], dtype=bool)
Unlike in many matrix languages, the product operator * operates elementwise in NumPy arrays. The matrix product can be performed using the dot function or creating matrix objects ( see matrix section of this tutorial ).
>>> A = array( [[1,1],
...             [0,1]] )
>>> B = array( [[2,0],
...             [3,4]] )
>>> A*B                         # elementwise product
array([[2, 0],
       [0, 4]])
>>> dot(A,B)                    # matrix product
array([[5, 4],
       [3, 4]])
%-----------------------------------------------------------------------------%
Some operations, such as += and *=, act in place to modify an existing array rather than create a new one.
\begin{verbatim}
>>> a = ones((2,3), dtype=int)
>>> b = random.random((2,3))
>>> a *= 3
>>> a
array([[3, 3, 3],
       [3, 3, 3]])
>>> b += a
>>> b
array([[ 3.69092703,  3.8324276 ,  3.0114541 ],
       [ 3.18679111,  3.3039349 ,  3.37600289]])
>>> a += b                                  # b is converted to integer type
>>> a
array([[6, 6, 6],
       [6, 6, 6]])
\end{verbatim}
%-----------------------------------------------------------------------------%
When operating with arrays of different types, the type of the resulting array corresponds to the more general or precise one (a behavior known as upcasting).
>>> a = ones(3, dtype=int32)
>>> b = linspace(0,pi,3)
>>> b.dtype.name
'float64'
>>> c = a+b
>>> c
array([ 1.        ,  2.57079633,  4.14159265])
>>> c.dtype.name
'float64'
>>> d = exp(c*1j)
>>> d
array([ 0.54030231+0.84147098j, -0.84147098+0.54030231j,
       -0.54030231-0.84147098j])
>>> d.dtype.name
'complex128'
Many unary operations, such as computing the sum of all the elements in the array, are implemented as methods of the ndarray class.
\begin{verbatim}
>>> a = random.random((2,3))
>>> a
array([[ 0.6903007 ,  0.39168346,  0.16524769],
       [ 0.48819875,  0.77188505,  0.94792155]])
>>> a.sum()
3.4552372100521485
>>> a.min()
0.16524768654743593
>>> a.max()
0.9479215542670073
\end{verbatim}
%-----------------------------------------------------------------------------%
By default, these operations apply to the array as though it were a list of numbers, regardless of its shape. However, by specifying the axis parameter you can apply an operation along the specified axis of an array:
>>> b = arange(12).reshape(3,4)
>>> b
array([[ 0,  1,  2,  3],
       [ 4,  5,  6,  7],
       [ 8,  9, 10, 11]])
>>>
>>> b.sum(axis=0)                            # sum of each column
array([12, 15, 18, 21])
>>>
>>> b.min(axis=1)                            # min of each row
array([0, 4, 8])
>>>
>>> b.cumsum(axis=1)                         # cumulative sum along each row
array([[ 0,  1,  3,  6],
       [ 4,  9, 15, 22],
       [ 8, 17, 27, 38]])
%--------------------------------------------------------------------------------------------%
\subsubsection{Universal Functions}
NumPy provides familiar mathematical functions such as sin, cos, and exp. In NumPy, these are called "universal functions"(ufunc). Within NumPy, these functions operate elementwise on an array, producing an array as output.
\begin{verbatim}
>>> B = arange(3)
>>> B
array([0, 1, 2])
>>> exp(B)
array([ 1.        ,  2.71828183,  7.3890561 ])
>>> sqrt(B)
array([ 0.        ,  1.        ,  1.41421356])
>>> C = array([2., -1., 4.])
>>> add(B, C)
array([ 2.,  0.,  6.])
\end{verbatim}

%-------------------------------------------------------------------------------%
\textbf{See also}\newline
all, alltrue, any, apply along axis, argmax, argmin, argsort, average, bincount, ceil, clip, conj, conjugate, corrcoef, cov, cross, cumprod, cumsum, diff, dot, floor, inner, inv, lexsort, max, maximum, mean, median, min, minimum, nonzero, outer, prod, re, round, sometrue, sort, std, sum, trace, transpose, var, vdot, vectorize, where
