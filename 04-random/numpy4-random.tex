\documentclass[numpymain.tex]{subfiles}
\begin{document}


\section{Random Numbers}

%http://www.engr.ucsb.edu/~shell/che210d/numpy.pdf


\begin{itemize}
\item An important part of any simulation is the ability to draw random numbers. 
\item For this purpose, 
we use \texttt{NumPy}'s built-in pseudorandom number generator routines in the sub-module 
random. 
\item The numbers are pseudo-random in the sense that they are generated 
deterministically from a seed number, but are distributed in what has statistical similarities to 
random fashion. 
\item NumPy uses a particular algorithm called the \textit{Mersenne Twister} to generate 
pseudorandom numbers. 
\end{itemize}

\subsection{Random Seeds}
The seed is an integer value. Any program that starts with the same seed will generate exactly 
the same sequence of random numbers each time it is run. This can be useful for debugging 
purposes, but one does not need to specify the seed and in fact, when we perform multiple 
runs of the same simulation to be averaged together, we want each such trial to have a 
different sequence of random numbers. If this command is not run, NumPy automatically 
selects a random seed (based on the time) that is different every time a program is run. 


\end{document}
