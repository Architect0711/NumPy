\documentclass[numpymain.tex]{subfiles}
\begin{document}


\section{Introduction to \texttt{numpy} }

%http://www.engr.ucsb.edu/~shell/che210d/numpy.pdf
%http://mentat.za.net/numpy/intro/intro.html

There are several ways to import \texttt{NumPy}. 
The standard approach is to use a simple \texttt{import} 
statement, additionally imported under the briefer name \texttt{np}:

\begin{framed}
\begin{verbatim}
import numpy as np
\end{verbatim}
\end{framed}

This statement will allow us to access NumPy objects using the syntax \texttt{np.object}.

%------------------------------------------------------ %

\subsection{Basic properties}
The NumPy arrays have three fundamental properties.

\begin{itemize}
\item \texttt{shape:} The shape attribute of any arrys describes its size
along all of its dimensions

\item \texttt{ndim:} The number of dimensions (often also called directions or axes)

\item \texttt{dtype:} The data type of the array elements.
\end{itemize}

\end{document}
